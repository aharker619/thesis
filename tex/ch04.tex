\chapter{Pathogenic bacteria can effect the endogenous actin cytoskeleton}\label{ch:vibrio}

\section[Abstract]{Abstract\footnotemark}
Vibrio cholera has multiple ways to hijack a cell’s actin cytoskeletal system
Toxins
Affect endogenous actin assembly proteins
Actin nucleators
VopL and VopF: two actin nucleators that assemble the endogenous actin into unproductive actin filaments
Understanding how vibrio cholera can hijack the endogenous actin system allows us to better fight the disease causing bacteria.

\footnotetext{Citations for chapter: [1] Thomas A. Burke, Alyssa J. Harker, Roberto Dominguez, and David R. Kovar. The bacterial virulence factors VopL and VopF nucleate actin from the pointed end. \textit{The Journal of Cell Biology}, March 2017. [2] Elena Kudryashova, David B. Heisler, Blake Williams, Alyssa J. Harker, Kyle Shafer, Margot E. Quinlan, David R. Kovar, Dimitrios Vavylonis, and Dmitri S. Kudryashov. Actin Cross-Linking Toxin Is a Universal Inhibitor of Tandem-Organized and Oliogmeric G-Actin Binding Proteins. \textit{Current biology}, 28{10}:1536-1547.e9, May 2018.}

\section{Introduction}\label{ch04-introduction}
Bacterial toxins can effectively compromise a host cell's functions with relatively few molecules, even leading to cell death. These toxins can target signaling cascades (ex. cGMP, adenylate cyclase) or inhibit other enzymes important for cellular processes such as protein synthesis \citep{henkel_toxins_2010}. As the actin cyctoskeleton is important for many cellular processes, it is commonly targeted by bacterial toxins. The actin cross-linking domain toxin (ACD) of \textit{Vibrio} species and related bacterial genera are deliverd to host cells by type 1 (MARTX toxin) or type VI (VgrG1 toxin) secretion systems. ACD catalyzes formation of actin oligomers through covalent crosslinking of Lys\textsuperscript{50} in subdomain 2 of an actin monomer with Glu\textsuperscript{270} in subdomain 3 of another actin monomer by an amide bond \citep{kudryashov_connecting_2008,kudryashova_glutamyl_2012}. This results in an oligomer that is not suitable for further actin polymerization because the two monomers are oriented similar to actin subunits along the short pitch of an actin filament, except that subdomain 2 has a major twist, disrupting the normal interface for further monomer binding \citep{kudryashov_connecting_2008}.  

However, there is a high concentration of actin yet only few ACD molecules secreted into the host cell. Using in vitro determined rates of ACD activity, it would take more than 6 months to covalently crosslink half of all the cytoplasmic actin with a single ACD molecule. This is beyond the timescale for in vivo measurements of monolayer disruption \citep{}. Previously, in a collaboration with the Kudryashov lab, we found that ACD is effective not by sequestering monomers as previously thought but by using actin oligomers to target formins. \citep{heisler_acd_2015}. We found that ACD formed toxic actin oligomers that blocked formin-mediated actin polymerization and nucleation. However, the mechanism of how these ACD-formed oligomers block formin activity remains unclear. 

Another way that Vibrio species target the actin cytoskeleton is through actin assembly proteins VopF or VopL. \citep{burke_bacterial_2017}

Actin nucleators: WH2 domains
	Controversy


\section{Results}\label{ch04-results}

\subsection{Ena/VASP is inhibited by actin crosslinking toxins}\label{ena-acd-oligomers}
Our previous study showed that formin mediated elongation of F-actin is blocked by the ACD oligomers. To further understand the mechanism of how these ACD oligomers affected actin nucleators we also tested Ena/VASP. We used single-molecule TIRF microscopy to measure the dynamics of single Ena/VASP proteins in the presence of increasing concentration of ACD oligomer in the presence of profilin. We found that Ena/VASP is affected by the ACD oligomers and will cap filaments, blocking growth. We calculated the percentage of capped filaments over a range of ACD oligomers and found that with increasing ACD oligomers, Ena/VASP capped more filaments. We also observed that the run length of Ena/VASP was much longer as a cap than as an elongation factor. 

\subsection{VopL and VopF assemble endogenous actin}\label{vops}

As a part of understanding how VopF and VopL can assemble endogenous actin we observed that in the presence of profilin and G-actin, VopF and VopL bind to the pointed end of actin filaments to nucleate actin filaments. VopF and VopL will bind for a certain amount of time, or residence time, after nucleating the actin filament. We measured the residence time of VopL and VopF to understand its dynamics on pointed barbed ends. Since these proteins are nucleating actin filaments from G-actin, we made two different residence time calculations. The first calculation is from the observed timepoint where an actin filament and Vop protein are visualized with TIRFM. The second calculation takes into account how long before the filament is able to be visualized due to the resolution of the TIRFM. Overall we observe that both VopL and VopF bind to the pointed end of filaments for similar amounts of times. 

