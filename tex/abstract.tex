\abstract
Many important cellular functions depend on the architecture of a dynamic actin cytoskeleton forming at the correct location and time during the cell cycle. Cellular division, motility, and endocytosis are a few examples of processes in which actin filaments must be nucleated, polymerized, severed, and depolymerized with spatiotemporal precision in the cell. To create a complete picture of the dynamic actin cytoskeleton, it is vital to collect mechanistic information about the diverse actin networks and their associated proteins. Actin networks gain their distinct properties from the subset of hundreds of actin binding proteins that localize to them. Some of these actin binding proteins, such as formin, have well-studied mechanisms of how the single molecule interacts with actin. However, how these proteins interact with other actin binding proteins and how these interactions affect activity to lead to complex actin networks has only recently become a topic of interest. Nonetheless, it is important that we first discern the molecular mechanisms of actin binding proteins on their own as this gives us a base mechanism to guide further experimentation. Understanding the molecular mechanistic details of actin binding proteins allows us to determine how they can function collectively to create, maintain, and disassemble complex actin networks. 

Here I investigate the molecular mechanisms of various actin binding proteins and how they are affected by the presence of other actin binding proteins and actin binding toxins. The main project focuses on Ena/VASP, which are tetrameric actin elongation factors that bind F-actin barbed ends continuously while increasing their elongation rate within dynamic bundled networks such as filopodia. However, we were also interested in how Ena/VASP's molecular mechanism is affected by the presence of different bundling proteins. We used single-molecule TIRFM and developed a kinetic model to dissect Ena/VASP’s processive mechanism on bundled filaments. The results demonstrate that Ena tetramers are tailored for enhanced processivity on fascin bundles and avidity of multiple arms associating with multiple filaments is critical for this process. 

I also was involved in many collaborations to investigate the dynamics and molecular mechanisms of various actin binding proteins. Many of these projects also surround the bundling proteins that I have studied in relation to Ena/VASP. I found that fascin, which is the main bundling proteins in filopodia and an enhancer of Ena/VASP processivity, also plays a role in reducing Arp2/3 complex-mediated branching. We also showed that fascin sorts with $\alpha$-actinin due to an intrinsic sorting mechanism dictated by filament spacing. We visualized this sorting by electron microscopy and observed the transition between a fascin domain and an $\alpha$-actinin domain. Further investigation into $\alpha$-actinin's bundling properties showed that tropomyosin increases $\alpha$-actinin dynamics. My other collaborations focused on the pathogenic \textit{Vibrio} bacteria. We found that an excreted toxin formed actin oligomers that inhibited Ena/VASP elongation and caused Ena/VASP to cap filaments. Furthermore, we investigated the molecular mechanism of a \textit{Vibrio} nucleation factor VopL/F and found that they nucleate filaments from the pointed end of F-actin and stay bound at the pointed end before releasing the filament. 

Overall, my work has shown the importance of not only characterizing the molecular mechanism of actin binding proteins, but also how these actin binding proteins work in concert with the multiple other actin binding proteins that are found within the cell. The gained knowledge from these studies are a step forward for the field to fully grasp the role each actin binding protein plays in the larger system of the actin cytoskeleton. 
