\abstract
Many important cellular functions depend on the architecture of a dynamic actin cytoskeleton forming at the correct location and time during the cell cycle. Cellular division, motility, and endocytosis are a few examples of processes in which actin filaments must be nucleated, polymerized, severed, and depolymerized at the precise time and location in the cell. 

To create a complete picture of the dynamic actin cytoskeleton, it is vital to collect mechanistic information about distinct actin networks and their associated proteins. The polymerization of specific actin networks at the precise time and location within a cell is important for many necessary cellular processes. Understanding the mechanistic details of individual actin binding proteins allow us to determine how they can function collectively to create, maintain and disassemble complex actin networks. Formins and Ena/VASP are two families of actin binding proteins that can bind to the barbed end of actin filaments and affect polymerization. Here we focus on the recent mechanistic findings of how these two families of proteins function individually and concurrently within the cell.
